
Some characteristics of the inverse Compton scattering (ICS) model
are reviewed. At least the following properties of radio pulsars
can be reproduced in the model: core or central emission beam, one
or two hollow emission cones, different emission heights of these
components, diverse pulse profiles at various frequencies, linear
and circular polarization features of core and cones.

We use cosmological N-body/gasdynamical simulations that include star formation
and feedback to examine the proposal that scaling laws between the total
luminosity, rotation speed, and angular momentum of disk galaxies reflect
analogous correlations between the structural parameters of their surrounding
dark matter halos.  The numerical experiments follow the formation of
galaxy-sized halos in two Cold Dark Matter dominated universes: the standard
$\Omega=1$ CDM scenario and the currently popular $\Lambda$CDM model.  We find
that the slope and scatter of the I-band Tully-Fisher relation are well
reproduced in the simulations, although not, as proposed in recent work, as a
result of the cosmological equivalence between halo mass and circular velocity:
large systematic variations in the fraction of baryons that collapse to form
galaxies and in the ratio between halo and disk circular velocities are observed
in our numerical experiments. The Tully-Fisher slope and scatter are recovered
in this model as a direct result of the dynamical response of the halo to the
assembly of the luminous component of the galaxy. We conclude that models that
neglect the self-gravity of the disk and its influence on the detailed structure
of the halo cannot be used to derive meaningful estimates of the scatter or
slope of the Tully-Fisher relation. Our models fail, however, to match the
zero-point of the Tully-Fisher relation, as well as that of the relation linking
disk rotation speed and angular momentum. These failures can be traced,
respectively, to the excessive central concentration of dark halos formed in the
Cold Dark Matter cosmogonies we explore and to the formation of galaxy disks as
the final outcome of a sequence of merger events. Disappointingly, our feedback
formulation, calibrated to reproduce the empirical correlations linking star
formation rate and gas surface density established by Kennicutt, has little
influence on these conclusions. Agreement between model and observations appears
to demand substantial revision to the Cold Dark Matter scenario or to the manner
in which baryons are thought to assemble and evolve into galaxies in
hierarchical universes.
